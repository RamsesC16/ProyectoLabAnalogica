\documentclass[journal]{IEEEtran}

% Paquetes base
\usepackage{instructivo}
\graphicspath{{./}{./fig/}}

% Figuras y utilidades
\usepackage[utf8]{inputenc}
\usepackage{graphicx}
\usepackage{subcaption}
\usepackage{array}
\usepackage{amsmath, amssymb}
\usepackage{siunitx}
\usepackage[hidelinks]{hyperref}
\usepackage{float} % para fijar figuras con [H]
\usepackage{url}
\usepackage{cite} % para [1], [2-3], etc.

\hyphenation{op-tical net-works semi-conduc-tor}
\sisetup{output-decimal-marker = {,}}

\begin{document}
	
	\title{Sistema de alarma por temperatura con indicadores luminosos, sirena y canal de audio}
	
\author{Santiago~Hernandez-Barrantes,~Axel~Castillo-Collao,~Julio ~ Quesada-Hernandez,~Ramses~Cortes-Torres,~\IEEEmembership{Estudiantes,~ITCR}%
		\\y Jos\'e~Miguel~Barboza-Retana,~\IEEEmembership{Profesor,~ITCR}}
	
	\markboth{EL3215 Proyecto de Electr\'onica Anal\'ogica, II-2025}{EL3215 Proyecto de Electr\'onica Anal\'ogica}
	
	\maketitle
	
	\begin{abstract}
		Se presenta el dise\~no e implementaci\'on de un sistema de alarma por temperatura utilizando dispositivos discretos y reguladores lineales. El sistema integra una fuente de alimentaci\'on desde 120~VAC, regulaci\'on a 5~V y 3,3~V, un subsistema de indicaci\'on luminosa tipo sem\'aforo gobernado por un sensor NTC, una sirena electr\'onica para condici\'on cr\'itica y una etapa de amplificaci\'on de audio micr\'ofono--parlante. Se describen los subsistemas implementados, la selecci\'on de componentes y la forma en que se interconectan, tomando como referencia bibliograf\'ia cl\'asica y aplicaciones pr\'acticas.
	\end{abstract}
	
	\begin{IEEEkeywords}
		Alarma de temperatura, BJT, NTC, sirena electr\'onica, reguladores lineales, amplificador de audio, sem\'aforo de LEDs.
	\end{IEEEkeywords}
	
	%------------------------------------------------------------
	\section{Introducci\'on}
	
	El proyecto tiene como objetivo implementar un sistema de alarma por temperatura que combine:
	\begin{itemize}
		\item indicaci\'on visual mediante tres estados luminosos asociados a distintos rangos de temperatura,
		\item una alerta sonora tipo sirena en condici\'on de sobretemperatura,
		\item y un canal de audio capaz de amplificar una se\~nal de baja amplitud hasta un parlante.
	\end{itemize}
	
	El dise\~no se realiza empleando exclusivamente componentes discretos (transistores BJT, diodos, resistencias y capacitores) y reguladores lineales simples, evitando microcontroladores o circuitos integrados complejos, en concordancia con los lineamientos del curso. Los criterios de polarizaci\'on y uso de BJT se apoyan en conceptos de textos de electr\'onica anal\'ogica \cite{razavi}, adaptados a topolog\'ias sencillas.
	
	A continuaci\'on se describen los subsistemas y su integraci\'on dentro del sistema completo.
	
	%------------------------------------------------------------
	
	\section{Diagrama general de sistema completo}
		
	\begin{figure}[H]
		\centering
		\includegraphics[width=\linewidth]{Diagrama}
		\caption{Diagrama de bloques}
		\label{fig:Diagrama}
	\end{figure}
	
	\section{Subsistema de fuente de alimentaci\'on}
	
	\subsection{Rectificaci\'on y filtrado}
	
	La alimentaci\'on parte de la red de 120~V\textsubscript{rms} a 60~Hz, modelada por una fuente senoidal y un transformador reductor T1. El secundario de T1 alimenta un puente rectificador de onda completa formado por diodos 1N4004 (D1--D4), configuraci\'on est\'andar para obtener tensi\'on continua a partir de CA \cite{psu_7805}.
	
	En cada semiciclo conducen dos diodos, por lo que la salida presenta una ca\'ida aproximada de \(2V_D\). El capacitor C1 de 100~\si{\micro\farad} se conecta en paralelo con la salida del puente, carg\'andose cerca del valor pico de la se\~nal rectificada y descarg\'andose entre picos, reduciendo el rizado.
	
	\begin{figure}[H]
		\centering
		\includegraphics[width=\linewidth]{fuente_rectificador}
		\caption{Rectificaci\'on de onda completa y filtrado capacitivo desde 120~VAC.}
		\label{fig:fuente_rectificador}
	\end{figure}
	
	\subsection{Regulaci\'on a 3,3 V y 5 V}
	
	Desde la tensi\'on continua filtrada se generan dos rieles:
	
	\begin{itemize}
		\item \textbf{3,3~V}: mediante un regulador LM1117-3.3, con capacitores de entrada y salida seg\'un hoja de datos, destinado a la etapa de amplificaci\'on micr\'ofono--parlante.
		\item \textbf{5~V}: mediante un regulador LM7805, con capacitores de desacople y filtrado \cite{7805_basico}, que alimenta el indicador luminoso y la sirena electr\'onica.
	\end{itemize}
	
	\begin{figure}[H]
		\centering
		\includegraphics[width=\linewidth]{reguladores}
		\caption{Implementaci\'on de los reguladores LM1117-3.3 y LM7805.}
		\label{fig:reguladores}
	\end{figure}
	
	%------------------------------------------------------------
	\section{Subsistema de indicaci\'on luminosa por temperatura}
	
	\subsection{Descripci\'on general}
	
	El subsistema de indicaci\'on luminosa implementa un comportamiento tipo sem\'aforo a partir de:
	\begin{itemize}
		\item un termistor NTC como sensor de temperatura,
		\item tres transistores NPN 2N2222A (Q3, Q4, Q5),
		\item potenciómetros para fijar umbrales,
		\item y LEDs con sus resistencias limitadoras.
	\end{itemize}
	
	La red NTC--resistencias genera una tensi\'on dependiente de la temperatura. Cada potenciómetro define un umbral distinto de disparo para cada transistor, siguiendo el principio de comparadores discretos con BJT utilizado en circuitos de indicador \cite{ntc_indicador,ntc_bc547}.
	
	\subsection{Operaci\'on}
	
	Cada rama Q--LED act\'ua como comparador de nivel:
	\begin{itemize}
		\item a baja temperatura, s\'olo se mantiene activo el estado seguro;
		\item al aumentar la temperatura y superar el primer umbral, conduce Q3 y se enciende el primer LED;
		\item al superar umbrales sucesivos, conducen Q4 y Q5, activando LEDs asociados a advertencia alta y alarma.
	\end{itemize}
	
	\begin{figure}[H]
		\centering
		\includegraphics[width=\linewidth]{indicador_ntc}
		\caption{Indicador luminoso gobernado por NTC y transistores 2N2222A.}
		\label{fig:indicador_ntc}
	\end{figure}
	
	%------------------------------------------------------------
	\section{Subsistema de sirena electr\'onica}
	
	\subsection{Topolog\'ia}
	
	La sirena se basa en un dise\~no cl\'asico de cuatro transistores:
	\begin{itemize}
		\item una primera etapa RC genera una se\~nal de baja frecuencia (oscilador lento),
		\item una segunda etapa produce el tono de audio,
		\item y una etapa final entrega potencia al parlante.
	\end{itemize}
	
	La topolog\'ia utilizada es consistente con propuestas de sirenas discretas documentadas en recursos t\'ecnicos \cite{unicrom_sirena,abcdatos_sirena}, adaptadas a los componentes disponibles.
	
	\subsection{Funcionamiento}
	
	Los capacitores de la etapa lenta se cargan y descargan alternadamente, modulando el punto de operaci\'on de la etapa osciladora r\'apida. Como resultado, la frecuencia del tono sube y baja peri\'odicamente, generando el efecto caracter\'istico de sirena.
	
	La etapa de salida utiliza un transistor de mayor capacidad de corriente para excitar un parlante de 8~\si{\ohm}. Este bloque se alimenta desde 5~V y se habilita solo cuando el sistema de indicaci\'on se encuentra en condici\'on de alarma.
	
	\begin{figure}[H]
		\centering
		\includegraphics[width=\linewidth]{sirena_4transistores}
		\caption{Sirena electr\'onica discreta con cuatro transistores.}
		\label{fig:sirena}
	\end{figure}
	
	%------------------------------------------------------------
	\section{Subsistema de amplificaci\'on micr\'ofono--parlante}
	
	\subsection{Descripci\'on del circuito}
	
	El canal de audio se implementa como un amplificador discreto alimentado a 3,3~V:
	\begin{itemize}
		\item Q1 (2N2222A) en emisor com\'un como preamplificador,
		\item C4 como capacitor de acople de entrada,
		\item Q2 (2N3904) como etapa de salida de baja impedancia,
		\item C7 como capacitor de acople hacia el parlante de 8~\si{\ohm}.
	\end{itemize}
	
	\subsection{Operaci\'on}
	
	La se\~nal del micr\'ofono o generador se acopla mediante C4 a la base de Q1, donde se amplifica en tensi\'on. Las variaciones en el colector de Q1 controlan a Q2, encargado de suministrar la corriente necesaria al parlante. El capacitor C7 bloquea la componente de continua.
	
	\begin{figure}[H]
		\centering
		\includegraphics[width=\linewidth]{amplificador_audio}
		\caption{Etapa de amplificaci\'on discreta micr\'ofono--parlante.}
		\label{fig:amplificador_audio}
	\end{figure}
	
	%------------------------------------------------------------
	\section{Integraci\'on de los subsistemas}
	
	\begin{itemize}
		\item La fuente rectificada y regulada genera los rieles de 5~V y 3,3~V.
		\item El subsistema con NTC define el estado de los LEDs seg\'un la temperatura.
		\item Al alcanzarse el umbral cr\'itico, la condici\'on de alarma habilita la sirena electr\'onica.
		\item El amplificador micr\'ofono--parlante opera sobre el riel de 3,3~V, aislado de las corrientes de la sirena y del indicador.
	\end{itemize}
	
	%------------------------------------------------------------
	\section{Conclusiones}
	
	El sistema desarrollado demuestra la viabilidad de implementar una alarma por temperatura completa utilizando elementos discretos y reguladores lineales sencillos. La fuente de alimentaci\'on proporciona tensiones estables desde 120~VAC; el indicador con NTC y transistores 2N2222A permite definir rangos de temperatura configurables; la sirena de cuatro transistores genera una se\~nal ac\'ustica claramente identificable; y la etapa micr\'ofono--parlante ejemplifica la amplificaci\'on de se\~nales de baja amplitud.
	
	%------------------------------------------------------------
	
	\begin{thebibliography}{9}
		
		\bibitem{razavi}
		B.~Razavi,
		\textit{Design of Analog CMOS Integrated Circuits},
		2nd ed., McGraw-Hill, 2016.
		
		\bibitem{unicrom_sirena}
		Unicrom,
		``Sirena con 4 transistores.''
		Disponible en:
		\url{https://unicrom.com/sirena-con-4-transistores/}.
		Accedido: 7 nov. 2025.
		
		\bibitem{abcdatos_sirena}
		ABCDatos,
		``Sirena con 4 transistores.''
		Disponible en:
		\url{https://www.abcdatos.com/tutoriales/tutorial/l9217.html}.
		Accedido: 7 nov. 2025.
		
		\bibitem{psu_7805}
		Easy Electronics Project,
		``5V Power Supply Circuit using 7805 Regulator.''
		Disponible en:
		\url{https://easyelectronicsproject.com/mini-projects/5v-power-supply-circuit-7805/}.
		Accedido: 29 oct. 2025.
		
		\bibitem{7805_basico}
		Electronics For You,
		``7805 Voltage Regulator: Pinout, Circuit and Uses.''
		Disponible en:
		\url{https://www.electronicsforu.com/technology-trends/learn-electronics/7805-ic-voltage-regulator}.
		Accedido: 29 oct. 2025.
		
		\bibitem{ntc_indicador}
		Homemade Circuits,
		``Simple Temperature Indicator Circuits using Thermistors.''
		Disponible en:
		\url{https://www.homemade-circuits.com/simple-temperature-indicator-using-thermistors/}.
		Accedido: 1 nov. 2025.
		
		\bibitem{ntc_bc547}
		A2AHelp,
		``Temperature Sensor Circuit Using Thermistor and BC547.''
		Disponible en:
		\url{https://a2ahelp.com/temperature-sensor-circuit-using-thermistor-and-bc547/}.
		Accedido: 1 nov. 2025.
		
		\bibitem{bjt_amp}
		EE Times,
		``Discrete audio amplifier basics -- Part 1: Bipolar junction transistor circuits.''
		Disponible en:
		\url{https://www.eetimes.com/discrete-audio-amplifier-basics-part-1-bipolar-junction-transistor-circuits/}.
		Accedido: 30 oct. 2025.
		
	\end{thebibliography}
	
\end{document}
